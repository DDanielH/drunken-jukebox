\section{Webservice}
\label{cha:Webservice}
Für den Zugriff auf die Daten stellen wir einen Webservice zur Verfügung, der aus folgenden REST-Services besteht:

\begin{description}
\item [Admin-Service] Stellt Funktionen zur Verwaltung der Party und der Songsammlung bereit.
\item [Party-Service] Stellt Funktionen für den Party-People bereit.
\end{description}


\subsection{Grundlegendes}
Im Folgenden werden zunächst grundlegende Objekte zur besseren Verständlichkeit der Schnittstelle erläutert.

\subsubsection{SongDTO}
\label{object:SongDTO}
Das Data-Transfer-Object beinhaltet zur Repräsentation eines Songs folgende Objekte:
\begin{itemize}
\item song: Enthält sämtliche Informationen zu einem \nameref{sec:Song}
\item{youtube: Link zum YouTube Video}
\item{localFile: Pfad zur lokalen Datei}
\end{itemize}

Im JSON-Format sieht ein SongDTO beispielsweise wie folgt aus:

\begin{lstlisting}[language=json]
    {
        "song": {
            "title": "Last Christmas",
            "interpret": "Wham!",
            "album": "",
            "durationInSecs": 277,
            "genres": []
        },
        "youtube": {
            "url": "https://www.youtube.com/watch?v=E8gmARGvPlI"
        },
        "localFile": null
    }
\end{lstlisting}

\subsubsection{VoteDTO}
\label{object:VoteDTO}
Das Data-Transfer-Object beinhaltet zur Repräsentation eines Votes folgende Objekte:
\begin{itemize}
\item{up: Boolscher Wert, der angibt, ob es sich um ein Up-Vote oder Down-Vote handelt.}
\item{songId: Id des Songs, für den das Voting bestimmt ist.}
\end{itemize}

Im JSON-Format sieht ein VoteDTO beispielsweise wie folgt aus:

\begin{lstlisting}[language=json]
    {
        "up": True
        "songId": 4711
    }
\end{lstlisting}

\subsection{AdminService}

\subsubsection{GetSongs}
\label{service:GetSongs}
Gibt alle Songs zur Songsammlung zurück.

GET
\url{/rest/dj/admin/songs}

Rückgabewert: JSON-formatierte Liste von SongDTOs 


\subsubsection{AddSongs}
\label{service:AddSongs}
Fügt eine Liste von Songs zur Songsammlung hinzu.

POST
\url{/rest/dj/admin/addSongs}

HTTP-Body: JSON-formatierte Liste von \nameref{object:SongDTO}.

\subsubsection{CreateSong}
\label{service:CreateSong}
Legt einen neuen Song anhand der übergebenen Informationen an und fügt diesen zur Songsammlung hinzu.

POST
\url{/rest/dj/admin/songs}

HTTP-Body: JSON-formatiertes \nameref{object:SongDTO}.


\subsubsection{UpdateSong}
\label{service:UpdateSong}
Aktualisiert den Song mit der entsprechenden SongId anhand der übergebenen Songinformationen.

PUT
\url{/rest/dj/admin/songs/{id}}

HTTP-Body: JSON-formatiertes \nameref{object:SongDTO}.

Parameter:\\
\begin{tabularx}{\textwidth}{|l|X|}
\hline id & Id des Songs, der aktualisiert werden soll. \\ 
\hline 
\end{tabularx}

\subsubsection{DeleteSong}
\label{service:DeleteSong}
Löscht den Song mit der angegebenen Id.

DELETE
\url{/rest/dj/admin/songs/{id}}

Parameter:\\
\begin{tabularx}{\textwidth}{|l|X|}
\hline id & Id des Songs, der gelöscht werden soll. \\ 
\hline 
\end{tabularx}

\subsubsection{CreateParty}
\label{service:CreateParty}
Startet eine neue Party.

POST
\url{/rest/dj/admin/party}

\subsubsection{UpdateParty}
\label{service:UpdateParty}
Stoppt die laufende Party.

DELETE
\url{/rest/dj/admin/party}

\subsection{PartyService}

\subsubsection{GetPlaylist}
\label{service:GetPlaylist}
Gibt die aktuelle Playlist zurück.

GET
\url{/rest/dj/party/playlist}

Rückgabewert: JSON-formatierte Playlist

\subsubsection{GetCurrentSong}
\label{service:GetCurrentSong}
Gibt den aktuellen Song zurück.

GET
\url{/rest/dj/party/currentSong}

Rückgabewert: JSON-formatierter Song

\subsubsection{GetDIValues}
\label{service:GetDIValues}
Gibt eine Liste von DIValues des entsprechenden Party-People zurück.

GET
\url{/rest/dj/party/peoples/{partyPeopleId}/diValues}

Rückgabewert: JSON-formatierte Liste von DIValues.

Parameter:\\
\begin{tabularx}{\textwidth}{|l|X|}
\hline partyPeopleId & Id des Party-People, dessen DIValues zurückgegeben werden sollen. \\ 
\hline 
\end{tabularx}

\subsubsection{GetVotes}
\label{service:GetVotes}
Gibt eine Liste von Votes des entsprechenden Party-People zurück.

GET
\url{/rest/dj/party/peoples/{partyPeopleId}/votes}

Rückgabewert: JSON-formatierte Liste von Vote.

Parameter:\\
\begin{tabularx}{\textwidth}{|l|X|}
\hline partyPeopleId & Id des Party-People, dessen Votes zurückgegeben werden sollen. \\ 
\hline 
\end{tabularx}

\subsubsection{RegisterPartyPeople}
\label{service:RegisterPartyPeople}
Fügt den Party-People der Party hinzu.

POST
\url{/rest/dj/party/peoples}

Rückgabewert: JSON-formatierter Party-People.

\subsubsection{Vote}
\label{service:Vote}
Für den Party-People mit der entsprechenden ID wird ein Voting abgegeben.

POST
\url{/rest/dj/party/peoples/{partyPeopleId}/votes}

HTTP-Body: JSON-formatiertes \nameref{object:VoteDTO}.

Parameter:\\
\begin{tabularx}{\textwidth}{|l|X|}
\hline partyPeopleId & Id des Party-People, der gevotet hat. \\ 
\hline 
\end{tabularx}

\subsubsection{SendDi}
\label{service:SendDi}
Für den Party-People mit der entsprechenden Id wird ein DIValue hinzugefügt.

POST
\url{/rest/dj/party/peoples/{partyPeopleId}/diValues}

HTTP-Body: JSON-formatierter Integer-Wert, der den DIValue repräsentiert.

Parameter:\\
\begin{tabularx}{\textwidth}{|l|X|}
\hline partyPeopleId & Id des Party-People, der seinen DIValue mitteilt. \\ 
\hline 
\end{tabularx}