\section{Fazit}
Aus dem Projekt heraus ist eine Anwendung entstanden, in der die Projektidee realisiert wurde. Unter Verwendung des Anwendungsservers WildFly haben wir das Backend für die Drunken-Jukebox implementiert.\\
Basierend auf dem Datenmodell, das wir zu Beginn des Projekts, persistiert der Anwendungsserver entsprechende Daten in einer frei wählbaren Datenbank. Die Anwendungslogik befindet sich in den Sessionbeans, welche innerhalb der externen Schnittstellen verwendet werden. So haben wir die Event-Subscribe-Methodik anhand eines Topics umgesetzt und eine REST-Schnittstelle geschaffen, die benötigte Beans injectet. Mittels dieser Schnittstellen ist das Backend bestens für den Einsatz vorbereitet. 

Rückblickend auf das gesamte Projekt stellen wir fest, dass wir zu Beginn hohe Erwartungen an den Applikationsserver gehegt haben und von dessen konzeptioneller Idee begeistert waren. Leider entwickelte sich unser Gesamteindruck des Produktes WildFly während des Praxiseinsatzes gegenteilig. \\
Die Anbindung unterschiedlicher Datenbankmanagementsysteme funktionierte erwartungsgemäß gut. Unser Datenmodell lies sich dabei problemlos abbilden. Im darauf folgenden Schritt schwand unsere Zuversicht schon bei dem Versuch Beans in einem REST-Service zu injecten. Erst nach einer Änderung unserer Architektur ließen sich dieses und kleinere weitere Probleme lösen. Ebenfalls verlief der Einsatz des Konzepts zur Authentifizierung und Autorisierung alles andere als zufriedenstellend, so dass auch mit Abschluss dieses Projekts das Konzept noch nicht vollständig für alle Bereiche (Beans, Webservice und Topics) funktioniert.\\
Diese elementaren Probleme in Kombination mit Fehlermeldungen, die meist keine Aussagekraft besitzen, und einer mangelhaften Dokumentation des Applikationsserver WildFly haben unseren Gesamteindruck zusätzlich negativ geprägt. Trotz des umfangreichen Engagements aller Projektbeteiligten konnten nicht alle Probleme beseitigt werden.

In einem kurzen Ausblick lässt sich festhalten, dass zu aller erst einmal die noch vorhandenen Probleme rund um die Authentifizierung bzgl. des Topics gelöst werden müssen, damit das Backend einsatzfähig ist. In weiteren Schritten sollten die Anwendungen für die Endanwender entwickelt werden. So gilt es eine Smartphone-App für die PartyPeople und ein Webinterface für den Admin bzw. den DJ zu entwickeln. Darüber hinaus muss ein Player zum Abspielen der Musik entwickelt werden, der mit der laufenden Anwendung auf dem Applikationsserver kommuniziert.
Sind die aufgezählten Anwendungen ebenfalls entwickelt, beginnt die Testphase. Eine mögliche Idee wäre der erstmalige Test auf einer Veranstaltung der Hochschule. Sind alle elementaren Funktionen erfolgreich implementiert, lassen sich weitere Features umsetzen.