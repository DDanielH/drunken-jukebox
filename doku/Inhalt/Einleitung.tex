\section{Einleitung}
Dieses Projekt entstand im Rahmen des Moduls Middleware A im ersten Semester des Masterstudiengangs Verteilte Systeme. Unter der Aufgabenstellung der Realisierung einer beliebigen Idee unter Einsatz des Applikationsservers WildFly galt es die experimentelle Verwendung aller Funktionen wie zum Beispiel Authentifizierung umzusetzen.
Die Freiheitsgrade in der Ideenfindung ließen uns das Konzept der Drunken-Jukebox entwickeln.

\subsection{Projektidee}
Jeder kennt es: Auf der Party läuft nicht die gewünschte Musik und früher oder später befinden sich viele Gäste an der Musikquelle, um ihren favorisierten Song in die Playlist aufzunehmen. Besser wird die Party dadurch aus Sicht der Allgemeinheit eher selten. Mit unserer Idee wollen wir genau dieses Problem lösen.

Die musikalische Gestaltung einer Party wird zu gleichen Teilen in die Hand der Gäste als auch in die des Gastgebers gegeben. Unsere Anwendung übernimmt die Position des Organisators der Playlist. Eigens entwickelte Algorithmen sollen eine Auswahl an Titeln aus einer vom Gastgeber definierten Titelsammlung unter Berücksichtigung verschiedener Größen zusammenstellen. Die wichtigsten Größen sind der Betrunkenheitsgrad der Gäste der aktuellen Party sowie die Votings der jeweiligen Titel aus vorangegangenen Partys.

Die so entstandene Playlist wird den Partygästen auf ihren Smartphones zur Anzeige gebracht und lässt sich von diesen manipulieren, indem die jeweiligen Songs bewertet und so in ihrer Reihenfolge verändert werden können. Bei entsprechend schlechter Bewertung kann ein Song auf diesem Wege aus der Playlist entfallen. Darüber hinaus ist der Gast verpflichtet, seinen Betrunkenheitsgrad durch das Lösen verschiedener Aufgaben oder Ähnlichem zu übermitteln. Die Anwendung lernt folglich mit jeder Party dazu und wählt im Laufe der Zeit immer passendere Songs aus. 

Grundsätzlich wird den Partygästen dazu eine Smartphone-App zur Verfügung gestellt und für den Gastgeber ein Administrationswerkzeug geschaffen. Im Rahmen dieses Projektes soll das Backend der Anwendung Drunken-Jukebox entstehen. 