\section{Einleitung}
Im Rahmen des Moduls Middleware A im vergangenen Semester hatten wir die Idee der Drunken-Jukebox. Dabei handelt es sich um ein System zur automatisierten Musikgestaltung auf einer Party, bei der die Gäste durch Voten von Songs die Musik-Playlist und damit den als nächstes gespielten Song beeinflussen können. Nachdem im Modul Middleware A ein entsprechendes Backend entwickelt worden ist, besteht die Aufgabe im Modul Middleware B nun daraus entsprechende Client-Applikationen unter Benutzung des Google Web Toolkits zu entwickeln. 

Entsprechend der Idee aus Middleware A soll es drei Client-Applikationen geben, die es zu realisieren gilt. Wir beschränken uns in diesem Projekt auf die Anwendungen für den Administrator und für den Partygast. Die dritte Anwendung zum Abspielen der Musik, haben wir bereits rudimentär in der Veranstaltung Middleware A implementiert. Aufgrund diverser fortwährender Probleme mit dem Applikationsserver WildFly, werden die Anwendungen im Rahmen dieses Projekts auf Deployd \footnote{\url{http://deployd.com/}} als alternatives Backend zugreifen. Im Nachfolgenden wird dieses wie auch die Planung, das Design sowie die letztendliche Realisierung der Client-Anwendungen vorgestellt. Es galt möglichst viele verschiedene Facetten, dessen was in GWT möglich ist, zu nutzen.
