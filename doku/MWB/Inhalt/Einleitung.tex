\section{Einleitung}
Im Rahmen des Moduls Middleware A im ersten Semester hatten wir die Idee der Drunken-Jukebox. Dabei handelt es sich um ein System zur automatisierten Musikgestaltung auf einer Party, bei der die Gäste durch Voten von Songs die Musik-Playlist und damit den als nächstes zu spielenden Song beeinflussen können. Nachdem im Modul Middleware A ein entsprechendes Backend entwickelt worden ist, besteht die Aufgabe im Modul Middleware B nun daraus, entsprechende Client-Applikationen unter Verwendung des Google Web Toolkits (GWT) zu entwickeln. Dabei galt es möglichst viele verschiedene Facetten von GWT zu benutzen.

Entsprechend der Idee aus Middleware A soll es drei Client-Applikationen geben, die es zu realisieren gilt. Wir beschränken uns in diesem Projekt auf die Anwendungen für den Administrator und für den Partygast. Die dritte Anwendung zum Abspielen der Musik, haben wir bereits rudimentär in der Veranstaltung Middleware A implementiert. Aufgrund fortwährender Probleme mit dem Applikationsserver WildFly, werden die Anwendungen im Rahmen dieses Projekts mit Deployd\footnote{\url{http://deployd.com/}} als ein alternatives Backend implementiert. Im Verlauf dieser Dokumentation wird neben Deployd vor allem die Planung und die Realisierung der beiden Client-Anwendungen beschrieben.