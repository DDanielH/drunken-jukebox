\section{Einleitung}
Im Rahmen des Moduls Middleware A im ersten Semester hatten wir die Idee der Drunken-Jukebox. Dabei handelt es sich um ein System zur automatisierten Musikgestaltung auf einer Party, bei der die Gäste durch Voten von Songs die Musik-Playlist und damit den als nächstes zu spielenden Song beeinflussen können. Nachdem im Modul Middleware A ein entsprechendes Backend entwickelt worden ist, besteht die Aufgabe im Modul Middleware B nun daraus, entsprechende Client-Applikationen unter Verwendung des Google Web Toolkits (GWT) zu entwickeln.

Entsprechend der Idee aus Middleware A gibt es drei Client-Applikationen. Wir beschränken uns in diesem Projekt auf die Anwendungen für den Administrator und für den Partygast, da wir die dritte Anwendung zum Abspielen der Musik bereits rudimentär in der Veranstaltung Middleware A implementiert haben. Aufgrund fortwährender Probleme mit dem Applikationsserver WildFly, werden die Anwendungen im Rahmen dieses Projekts mit Deployd\footnote{\url{http://deployd.com/}} als ein alternatives Backend implementiert. Im Verlauf dieser Dokumentation wird neben Deployd vor allem die Planung und die Realisierung der beiden Client-Anwendungen beschrieben.

Aus technischer Sicht sollen möglichst viele verschiedene Facetten des GWT-Frameworks genutzt werden. Dazu zählen unter anderem die Verwendung von MVP (Model-View-Presenter) beim Softwaredesign und die Erstellung eigener Widgets bei der Implementierung der Views. Zudem planen wir den Einsatz von GSS (Google Closure Stylesheets) als Präprozessor für die Stylesheets und JSNI (JavaScript Native Interface) zum Einbinden von nativem JavaScript-Code. Als letztes sollen die Lokalisierungsmöglichkeiten von GWT zur Bereitstellung der Anwendungen in mehreren Sprachen genutzt werden.