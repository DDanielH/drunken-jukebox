\section{Realisierung}

In diesem Kapitel wird die konkrete Umsetzung der Anforderungen in GWT beschreiben.
Zuerst wird die Architektur der Anwendung festgelegt. Anschließend werden konkrete
Implentierungsdetails erläutert, welche die Erstellung eigener Widgets in GWT und
mittels JSNI, das Styling in GSS sowie die Lokalisierung.

\subsection{Architektur}

Um den Anforderungen der zwei Anwendergruppen (Veranstalter und Gäster einer Party) gerecht
zu werden, entwickeln wir zwei Anwendungen. Die Admin-Oberfläche wird als Desktop-Webanwendung
entwickelt, während die Vote-App als mobile Webanwendung umgesetzt wird.

Zwei GWT-Anwendungen
- eine für Desktop
- eine für mobil

MVP-Architektur
- Klassendiagramm für beide (Views \& Presenter)

Backend
- Java-Server als Proxy
- Deployd als Datenhaltung + Logik

Fabian

\subsection{Eigene Widgets}
- Search Dinge
- PlaylistEntries

Beispiele

Chris

\subsection{Styles in GSS}
- Getrennt für beide Anwendungen (Desktop und Mobil)

Fabian

\subsection{Lokalisierung}
- Admin: Wählen
- VoteApp: Automatisch

Fabian

\subsection{JSNI}
Slider mit jQuery

Daniel