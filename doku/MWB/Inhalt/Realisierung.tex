\section{Realisierung}

In diesem Kapitel wird die konkrete Umsetzung der Anforderungen in GWT beschreiben.
Zuerst wird die Architektur der Anwendung festgelegt. Anschließend werden konkrete
Implentierungsdetails erläutert, welche die Erstellung eigener Widgets in GWT und
mittels JSNI, das Styling in GSS sowie die Lokalisierung.

\subsection{Architektur}

Um den Anforderungen der zwei Anwendergruppen (Veranstalter und Gäster einer Party) gerecht
zu werden, entwickeln wir zwei Anwendungen. Die Admin-Oberfläche wird als Desktop-Webanwendung
entwickelt, während die Vote-App als mobile Webanwendung umgesetzt wird. Trotzdem werden
beide Anwendungen mit einer ähnlichen Softwarearchitektur implementiert.

TODO: Blockdiagramm der Architektur

\subsubsection{Frontend}
MVP-Architektur
- Klassendiagramm für beide (Views \& Presenter)

\subsubsection{Backend}
- Java-Server
- Als Proxy für Deployed
- Deployed zur Datenhaltung + Logik

Fabian

\subsection{Eigene Widgets}
- Search Dinge
- PlaylistEntries

Beispiele

Chris

\subsection{Styles in GSS}
Für das Styling der beiden GWT-Anwendungen verwenden wir GSS. Da es unterschiedliche Anforderungen
an Deskop- und Mobilanwendungen gibt, werden zwei Stylesheet-Dateien verwendet.

- Verknüpfung: Client-Bundle und GSS-Dateien
- Gemeinsame Styles auslagern?

Fabian

\subsection{Lokalisierung}
Beide Anwendungen sollen lokalisiert in Deutsch und Englisch angeboten werden.
Im Admin soll es möglich sein die Sprache über die Oberfläche zu ändern, während
in der Vote-App die Sprache automatisch ausgewählt wird.

Dazu verwenden wir Ressourcen-Strings (TODO).

Fabian

\subsection{JSNI}
Slider mit jQuery

Daniel