\section{GWT-RPC}

\subsection{Admin-Service}
Nachfolgend eine Liste aller Funktionen des Admin-Services:
\begin{description}
	\item[ArrayList<Song> getSongList()] Gibt die Liste aller Songs zurück.
	\item[Song getSong(String id)] Gibt einen spezifischen Song zurück.
	\item[Song updateSong(Song song)] Updatet einen Song.
	\item[void removeSong(String songId)]	Löscht einen Song.
	\item[Song addSong(Song song)] Fügt einen neuen Song hinzu.
	\item[Party startParty()]	Startet eine Party
	\item[Party stoppParty(Party p)] Stoppt eine Party
	\item[GlobalPlaylist getPlaylist()] Gibt die aktuelle Playlist zurück zurück.
\end{description}


\subsection{VoteApp-Service}
Nachfolgend die Auflistung der Funktionen des VoteApp-Services:
\begin{description}
	\item[Song getCurrentSong()] Gibt den aktuellen Song zurück.
	\item[PlayList getPlayList()]	Gibt die aktuelle Playlist zurück.
	\item[void sendDi(int value)] Sendet den übergebenen Drunken-Index.
	\item[void sendVote(PlayListEntry entry, Vote vote)] Sendet zum übergebenen Song (PlaylistEntry) einen Vote (Up oder Down).
\end{description}

\section{REST-Schnittstelle von Deployd}
\label{anh:rest}
Deployd stellt für den Zugriff auf die Entitäten eine REST-Schnittstelle zur Verfügung und unterstützt dabei standardmäßig die HTTP-Methoden GET, POST, PUT und DELETE\footnote{\url{http://docs.deployd.com/docs/collections/reference/http.html}}. Im Folgenden beschreiben wir die wichtigsten, in unserem Projekt verwendeten HTTP-Methoden.

\subsection{party}
\label{service:party}
Erstellt eine neue Party und löscht die Playlist.

POST
\url{/party}

Body: Leeres JSON-Objekt

Rückgabewert: JSON-Objekt der Party

Beispiel:
\begin{lstlisting}[language=Java]
{
  "start":"2015-06-03T12:12:48.190Z",
  "end":{},
  "avgDI":0,
  "guestCount":1,
  "id":"791bbdf86d85fa29"
}
\end{lstlisting}
Darüber hinaus ist auch ein DELETE möglich, welches die Party beendet.

\subsection{playlist}
\label{service:playlist}
Gibt die aktuelle Playlist zurück.

GET
\url{/playlist}

Rückgabewert: JSON-Array aller Playlist Einträge

Beispiel:
\begin{lstlisting}[language=Java]
[
{
  "songID":"89d0ddbf0d7e683e","position":0,"votes":1,"id":"b90bb3e138456920"
},
{
  "songID":"fbc56a0eacd2d880","position":0,"votes":2,"id":"7de1eb1dfeee188e"
},
...
]
\end{lstlisting}


\subsection{currentSong}
\label{service:currentSong}
Gibt den aktuellen Song zurück.

GET
\url{/currentsong}

Rückgabewert: JSON-Objekt des aktuellen Songs

Beispiel:
\begin{lstlisting}[language=Java]
{
  "songID":"18c77e30e376280f",
  "id":"77a26abb92f2ebdd"
}
\end{lstlisting}

\subsection{song}
\label{service:song}
Gibt alle Songs zurück.

GET
\url{/song}

Rückgabewert: JSON-Array aller Songs

Beispiel:
\begin{lstlisting}[language=Java]
[
{
  "title":"The Pretender",
  "length":260,
  "artist":"Foo Fighters",
  "source":"C:\\Musik\\",
  "sourceType":0,
  "genres":"Rock",
  "id":"89d0ddbf0d7e683e"
},
{
  "title":"Crawling",
  "length":340,
  "artist":"Linkin Park",
  "source":"C:\\Musik\\",
  "sourceType":0,
  "genres":"Rock",
  "id":"fbc56a0eacd2d880"
},
...
]
\end{lstlisting}