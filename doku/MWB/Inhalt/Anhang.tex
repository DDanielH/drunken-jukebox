\section{GWT-RPC}

Schnittstellen kurz beschreiben

Zwei Services:
- AdminService
	- getCurrentSong(): Liefert den aktuellen Song zurück
	- getPlaylist(): Gibt die aktuelle Playlist zurück
	- sendDi(int value): Schickt den DI ans Backend
	- sendVote(PlaylistEntry entry, Vote vote): Funktionen zum Voten eines Playlist-Eintrags...
- VoteAppService

\section{REST-Schnittstelle von Deployd}

\subsection{party}
\label{service:party}
Erstellt eine neue Party und löscht die playlist.

POST
\url{/party}

Parameter: Leeres JSON Objekt

Rückgabewert: JSON Objekt der Party
Beispiel:
\begin{lstlisting}
{
"start":"2015-06-03T12:12:48.190Z",
"end":{},
"avgDI":0,
"guestCount":1,
"id":"791bbdf86d85fa29"
}
\end{lstlisting}

\subsection{playlist}
\label{service:playlist}
Gibt die aktuelle Playlist zurück.

GET
\url{/playlist}

Rückgabewert: JSON Array aller Playlist Einträge
Beispiel:
\begin{lstlisting}
[
{"songID":"89d0ddbf0d7e683e","position":0,"votes":1,"id":"b90bb3e138456920"},
{"songID":"fbc56a0eacd2d880","position":0,"votes":2,"id":"7de1eb1dfeee188e"},
...
]
\end{lstlisting}


\subsection{currentSong}
\label{service:currentSong}
Gibt den current Song zurück.

GET
\url{/currentsong}

Rückgabewert: JSON Objekt des current Songs
Beispiel:
\begin{lstlisting}
[{
"songID":"18c77e30e376280f",
"id":"77a26abb92f2ebdd"
}]
\end{lstlisting}

\subsection{song}
\label{service:song}
Gibt alle Songs zurück.

GET
\url{/song}

Rückgabewert: JSON Array aller Songs
Beispiel:
\begin{lstlisting}
[{
"title":"The Pretender",
"length":260,
"artist":"Foo Fighters",
"source":"C:\\Musik\\",
"sourceType":0,
"genres":"Rock",
"id":"89d0ddbf0d7e683e"
},
{"title":"Crawling",
"length":340,
"artist":"Linkin Park",
"source":"C:\\Musik\\",
"sourceType":0,
"genres":"Rock",
"id":"fbc56a0eacd2d880"
},
...
]
\end{lstlisting}



Zudem wir noch POST, PUT, und DELETE unterstützt\footnote{\url{http://docs.deployd.com/docs/collections/reference/http.html}}.


