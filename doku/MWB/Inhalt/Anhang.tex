\section{GWT-RPC}

\subsection{Admin Service}
Nachfolgend eine Liste aller Funktionen des Admin-Services:
\begin{description}
	\item[ArrayList<Song> getSongList()] Gibt die Liste aller Songs zurück.
	\item[Song getSong(String id)] Gibt einen spezifischen Song zurück.
	\item[Song updateSong(Song song)] Updatet einen Song.
	\item[void removeSong(String songId)]	Löscht einen Song.
	\item[Song addSong(Song song)] Fügt einen neuen Song hinzu.
	\item[Party startParty()]	Startet eine Party
	\item[Party stoppParty(Party p)] Stoppt eine Party
	\item[GlobalPlaylist getPlaylist()] Gibt die aktuelle Playlist zurück zurück.
\end{description}


\subsection{VoteApp Service}
Nachfolgend die Auflistung der Funktionen des VoteApp-Services:
\begin{description}
	\item[Song getCurrentSong()] Gibt den Current Song zurück.
	\item[PlayList getPlayList()]	Gibt die aktuelle Playlist zurück.
	\item[void sendDi(int value)] Sendet den übergebenen Betrunkenheitsgrad (DI-Wert) zurück.
	\item[void sendVote(PlayListEntry entry, Vote vote)] Sendet zum Übergebenen Song (PlaylistEntry) einen Vote (UP oder DOWN).
\end{description}

\section{REST-Schnittstelle von Deployd}

\subsection{party}
\label{service:party}
Erstellt eine neue Party und löscht die playlist.

POST
\url{/party}

Parameter: Leeres JSON Objekt

Rückgabewert: JSON Objekt der Party
Beispiel:
\begin{lstlisting}
{
"start":"2015-06-03T12:12:48.190Z",
"end":{},
"avgDI":0,
"guestCount":1,
"id":"791bbdf86d85fa29"
}
\end{lstlisting}

\subsection{playlist}
\label{service:playlist}
Gibt die aktuelle Playlist zurück.

GET
\url{/playlist}

Rückgabewert: JSON Array aller Playlist Einträge
Beispiel:
\begin{lstlisting}
[
{"songID":"89d0ddbf0d7e683e","position":0,"votes":1,"id":"b90bb3e138456920"},
{"songID":"fbc56a0eacd2d880","position":0,"votes":2,"id":"7de1eb1dfeee188e"},
...
]
\end{lstlisting}


\subsection{currentSong}
\label{service:currentSong}
Gibt den current Song zurück.

GET
\url{/currentsong}

Rückgabewert: JSON Objekt des current Songs
Beispiel:
\begin{lstlisting}
[{
"songID":"18c77e30e376280f",
"id":"77a26abb92f2ebdd"
}]
\end{lstlisting}

\subsection{song}
\label{service:song}
Gibt alle Songs zurück.

GET
\url{/song}

Rückgabewert: JSON Array aller Songs
Beispiel:
\begin{lstlisting}
[{
"title":"The Pretender",
"length":260,
"artist":"Foo Fighters",
"source":"C:\\Musik\\",
"sourceType":0,
"genres":"Rock",
"id":"89d0ddbf0d7e683e"
},
{"title":"Crawling",
"length":340,
"artist":"Linkin Park",
"source":"C:\\Musik\\",
"sourceType":0,
"genres":"Rock",
"id":"fbc56a0eacd2d880"
},
...
]
\end{lstlisting}



Zudem wir noch POST, PUT, und DELETE unterstützt\footnote{\url{http://docs.deployd.com/docs/collections/reference/http.html}}.


