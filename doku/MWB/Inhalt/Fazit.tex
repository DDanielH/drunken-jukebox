\section{Fazit}
Im Verlauf des MWB-Projekts wurden zwei Anwendungen in GWT implementiert, die mit dem alternativen Backend auf Basis von Deployd kommunizieren. Die Anwendung für den Administrator erfüllt ebenso wie auch die für die Partygäste alle eingangs spezifizierten Anforderungen. Im Zuge vom Einsatz von MVP haben wir zahlreiche eigene Widgets entwickelt, die in beiden Anwendungen genutzt werden. Die Vorteile von MVP sind allerdings aufgrund des Umfangs unseres Projekts leider nicht deutlich geworden. Positiv überrascht hingegen waren wir im Vergleich zum Schwerpunkt des Moduls MWA, WildFly, wie gut GWT, der Schwerpunkt dieses Moduls, MWB, funktioniert. So brachte beispielsweise ein Upgrade der GWT-Version von 2.6 auf 2.7 zur Halbzeit unseres Projekts keinerlei Probleme mit sich und eine sehr große Auswahl der wichtigsten Oberflächenkomponenten stehen bereits im Standard zur Verfügung. Darüber hinaus verlief der Einsatz von JSNI (gespr. Jizznieh) ohne erwähnenswerte Komplikationen. Ebenso gut hat uns das Grundprinzip von GSS gefallen, wobei sich teilweise der Eindruck verstärkt hat, dass es nicht zu Ende gedacht ist. Zuweisungen und das Durchreichen von Styles ist gewöhnungsbedürftig und wir hätten uns gewünscht, dass es möglich ist mehrere CSS-Definitionen auf einer höheren Abstraktionsebene zu überspannen.

Insgesamt hat das Projekt gezeigt, dass GWT ein gutes Tool ist, welches zum Einsatz in produktiven Umgebungen geeignet ist.
