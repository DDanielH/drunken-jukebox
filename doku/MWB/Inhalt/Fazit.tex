\section{Fazit}
Im Verlauf des MWB-Projekts wurden zwei Anwendungen in GWT implementiert, die über einen Proxy mit dem Backend auf Basis von Deployd als Alternative zum WildFly kommunizieren. Sowohl die Anwendung für den Administrator als auch die App für die Partygäste erfüllen alle eingangs spezifizierten Anforderungen. Unter der Verwendung von MVP haben wir zahlreiche eigene Widgets entwickelt, die in beiden Anwendungen genutzt werden. Die Vorteile von MVP sind allerdings aufgrund der geringen Komplexität unseres Projekts nicht deutlich geworden. 

Die Verwendung von GWT und den dazu gehörigen Erweiterungen verlief für ein so großes Framework erstaunlicherweise problemlos.
So brachte beispielsweise ein Upgrade der GWT-Version von 2.6 auf 2.7 zur Mitte unseres Projekts keinerlei Probleme mit sich. Zudem stehen eine sehr große Auswahl der wichtigsten Oberflächenkomponenten bereits im Standard zur Verfügung. Darüber hinaus verlief der Einsatz von JSNI  ohne erwähnenswerte Komplikationen. Ebenso gut hat uns das Grundprinzip von GSS gefallen, wobei sich teilweise der Eindruck verstärkt hat, dass es nicht zu Ende gedacht ist. Zuweisungen und das Durchreichen von Styles ist gewöhnungsbedürftig und wir hätten uns gewünscht, dass es möglich ist, mehrere CSS-Definitionen auf einer höheren Abstraktionsebene zu überspannen.

Insgesamt hat das Projekt gezeigt, dass GWT ein gutes Tool ist, welches zum Einsatz in produktiven Umgebungen geeignet ist. Wir sind abschließend mit unserem Projekt und den Möglichkeiten von GWT zufrieden.
