\section{Fazit}
Aus dem Projekt heraus ist eine Anwendung entstanden, die der ursprünglichen Projektidee sehr nahe kommt. Mit Hilfe des Anwendungsservers WildFly entstand das Backend für die \textit{Drunken-Jukebox}, das alle wesentlichen Funktionen implementiert.\\
Auf Basis des zu Beginn des Projekts entwickelten Datenmodells persistiert der Anwendungsserver entsprechende Daten in einer frei wählbaren Datenbank. Die Anwendungslogik befindet sich in den Sessionbeans, welche eng verknüpft mit den externen Schnittstellen kooperieren. So haben wir die Event-Subscribe-Methodik anhand eines Topics umgesetzt und eine REST-Schnittstelle geschaffen, die benötigte Beans injectet. Mittels dieser Schnittstellen ist das Backend bestens für den Einsatz vorbereitet. 

Bei der Reflektion über die gesamte Zeitspanne des Projekt hinweg stellen wir fest, dass wir zu Beginn hohe Erwartungen an den Applikationsserver gehegt haben und von dessen konzeptioneller Idee begeistert waren. Leider entwickelte sich unser Gesamteindruck des Produktes WildFly während des Praxiseinsatzes gegenteilig. \\
Während wir anfangs noch relativ problemlos den EntityManager einsetzten und dieser reibungslos mit diversen Datenbanken kooperierte, schwand unsere Zuversicht schon bei dem Versuch Beans in einem REST-Service zu injecten. Erst nach einer Änderung unserer Architektur ließen sich dieses und kleinere weitere Probleme lösen. Ebenfalls verlief der Einsatz des Konzepts zur Authentifizierung und Autorisierung alles andere als zufriedenstellend, so dass auch mit Abschluss dieses Projekts das Konzept noch nicht vollständig für alle Bereiche (Beans, Webservice, Queues/Topics) funktioniert.\\
Diese elementaren Probleme in Kombination mit Fehlermeldungen, die meist keine Aussagekraft besitzen, und einer schlechten Dokumentation des Applikationsserver WildFly haben unseren Gesamteindruck über die gesamte Zeit eher negativ geprägt. Zu unserer Projektarbeit lässt darüber hinaus reflektierend sagen, dass wir ein sehr ausgeglichenes und motiviertes Team waren. 

In einem kurzen Ausblick lässt sich festhalten, dass zu aller erst einmal die noch vorhandenen Probleme rund um die Authentifizierung bzgl. des Topics gelöst werden müssen, damit das Backend einsatzfähig ist. In weiteren Schritten sollten die Anwendungen für die Endanwender entwickelt werden. So gilt es eine Smartphone-App für die PartyPeople und ein Webinterface für den Admin bzw. den DJ zu entwickeln. Darüber hinaus muss ein Player zum Abspielen der Musik entwickelt werden, der mit der laufenden Anwendung auf dem Applikationsserver kommuniziert.
Sind die aufgezählten Anwendungen ebenfalls entwickelt, beginnt die Testphase. Eine mögliche Idee wäre der erstmalige Test auf einer Veranstaltung der Hochschule. Sind alle elementaren Funktionen erfolgreich implementiert, lassen sich weitere Features umsetzen. In dem Fall sind der Kreativität keine Grenzen gesetzt.